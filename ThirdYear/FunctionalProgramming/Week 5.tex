\documentclass{article}
\usepackage{amsmath}
\usepackage{xcolor}
\usepackage{graphicx}
\graphicspath{ {./images/} }
\title{Functional Programming - Week 5}
\author{Varjak Wolfe, 18325326}
\begin{document}

\textbf{Classes Based on Other Classes}
"=>" is context, stating that the second class depends on the first

\textbf{Polymorphic Type Classes}

How might we define an Eq instance for lists?

\begin{itemize}
\item For \textcolor{green}{[Bool]}
\textcolor{green}{instance Eq [Bool] where}

\textcolor{green}{[ ] == [ ] = True}

\textcolor{green}{(b1:bs1) == (b2:bs2) = b1 == b2 \&\& bs1 == bs2}

\textcolor{green}{\textunderscore == \textunderscore = False}

\item Cant we do this polymorphically?

\item We can!

\textcolor{green}{instance (Eq a) => Eq [a] where}

\textcolor{green}{[ ] == [ ] = True}

\textcolor{green}{(x1:xs1) == (x2:xxs2) = x1 ==x2 \&\& xs1 == xs2}

\textcolor{green}{\textunderscore == \textunderscore = False}
\end{itemize}

We can define equality on [a] provided we have equality set up for a

Here we are defining equlity for a type constructor ([ ] for lists) applied to a type a.


\textbf{How Haskell Handles a Class Name/Operator}

Consider the following polymorphic function:

\textcolor{green}{threewayEq  ::  Eq a => [a] -> [a] -> [a] -> Bool}

\textcolor{green}{threewayEq  xs  ys  ys = xs == ys \&\& ys == zs}

The code for threewayEq is polymorphic so at compilation time, we cant say which implementation of == is used. Bu tin the general case, a function like this gets an additional aprameter, a "class dictionary".

This is used at run-time to look up the implementation of ==, once a concrete instance for type a is known.


\textbf{Type-Constructor Classes}

Consider the class declaration for \textcolor{green}{Functor}

\textcolor{green}{class Functor  f where}

\textcolor{green}{fmap  ::  (a -> b) -> f  a -> f  b}

The idea here is that \textcolor{green}{fmap  f} applied to a value of type \textcolor{green}{f  a} will apply f to all occurences of a within that value.

Here, we are asssociating a class with a type-constructor  f


\textbf{Type-Constructor Examples}

The \textcolor{green}{Maybe} type-constructor

\textcolor{green}{data Maybe a = Nothing | Just a}

\textbf{Instances of Functor}

\textcolor{green}{Maybe} as a \textcolor{green}{Functor}

\textcolor{green}{instance Functor Maybe where}

\textcolor{green}{fmap f Nothing = Nothing}

\textcolor{green}{fmap f (Just a) = Just (f x)}


\textbf{Functor Instance for Maybe}

\textcolor{green}{class Functor f where}

\textcolor{green}{fmap :: (a -> b) -> f a -> f b}


\textcolor{green}{instance Functor Maybe where}

\textcolor{green}{fmap (f :: a -> b) (Nothing :: Maybe a) = Nothing :: Maybe b}

\textcolor{green}{fmap f (Just (x :: a) :: Maybe a) = Just (f x :: b) :: Maybe b}


\textbf{An Example: Expressions}

We are going to write functions that manipulate expressions in a variety of ways.

\includegraphics{expr}

$(10 + 5) * 90$ becomes \textcolor{green}{Mul  (Add  (Val 10)  (Val 5))  (Val 90)}

$10 + (5 * 90)$ becomes \textcolor{green}{Add  (Val 10)  (Mul  (Val 5)  (Val 90))}

We can write a function to calculate the result of these expressions:

\textcolor{green}{eval  ::  Expr -> Double}

\textcolor{green}{eval  (Val x) = x}

\textcolor{green}{eval  (Add  x  y) = eval x + eval y}

\textcolor{green}{eval  (Mul  x  y) = ...} --similar to above. do similarly for sub and dvd

\textcolor{red}{$>$ eval Add  (Val 10)  (Mul  (Val 5)  (Val 90))}

\textcolor{red}{460.0}

We can also write a function to simplify expressions

\textcolor{green}{simp  ::  Expr -> Expr}

\textcolor{green}{simp  (Val x) = (Val x)}

We can use pattern matching in let-expressions!

\textcolor{green}{simp (Add  e1  e2) = let  (Val x) = simp e1}

\textcolor{green}{(val y) = simp e2}

\textcolor{green}{in Val (x+y)}

simp  (Dvd  x  y) = ....


\textbf{Adding Variables to Expressions}

\includegraphics{var}

We can also see that our simplification will return either a \textcolor{red}{(Val value)} or a \textcolor{red}{(Var var)}.

\textcolor{green}{simp  (Var v) = (Var v}

This complicates our simp somewhat because we can no longer assume that simp always returns a Val.

\textbf{Simplification for Operators}

We now have to pattern-match on the results of recursive calls to simp

\includegraphics{simp1}

\includegraphics{simp2}


\textbf{Evaluating Exprs with Variables}

We can't fully evaluate our extended expression language without some way of knowing what values any of the variables (Var) have. We can imagine eval should have a signature like this:

\textcolor{red}{eval :: Dict Id Double -> Expr -> Double}

It now has a new (first) argument, a Dict that associates Double (data values) with Id (key values).


\textbf{How to model a lookup dictionary?}

A dictionary maps keys to data values. An obvious approach is to use a list of key/data pairs:

\textcolor{red}{type Dict k d = [ (k,d) ]}

Defining a link between key and data is simply cons-ing such a pair onto the start of the list:

\textcolor{red}{define :: Dict k d -> k -> d -> Dict k d}

\textcolor{red}{define d s v - (s,v):d}

Lookup simply searches along the list:

\textcolor{red}{find :: Eq k => Dict k d -> k -> Maybe d}

\textcolor{red}{find [ ] \textunderscore = Nothing}

\textcolor{red}{find ( (s,v) : ds) name | name == s = Just v}

\textcolor{red}{  				|otherwise = find ds name}

We need to handle the case when the key is not present, using the Maybe type





\textbf{Expressions with Local Variable Declarations}



\includegraphics{extended}

\textcolor{green}{Def  x  e1  e2} means: x is in scope in e2, bu tno in e1; compute value of e1, and assign value to x; then evaluate e2 as the overall result


\textbf{Abstracting Functions}

Consider the following function definitions:

\textcolor{green}{f  a  b = sqr a + sqrt b}

\textcolor{green}{g  x  y = sqrt x * sqr y}

They all have the same general form:

\textcolor{green}{fname arg1 arg2 = someF arg1 'someOp' anotherF arg2}

We can abstract this by adding parameters to represent the bits of the general form.

\textcolor{green}{absF someF anotherF someOp arg1 arg2 = omeF arg1 'someOp' anotherF arg2}

Now f and g can be defined using absF:

\textcolor{green}{f  a  b = absF sqr sqrt (+) a b}

\textcolor{green}{g  x  y = absF sqrt sqr (*) x y}



\end{document}

