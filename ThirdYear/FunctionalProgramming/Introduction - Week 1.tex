\documentclass{article}
\title{Functional Programming - Week 1}
\usepackage{amsmath}
\usepackage{graphicx}
\graphicspath{ {./images/} }
\author{Varjak Wolfe, 18325326}
\begin{document}
\textbf{Defining Haskell Values}


\begin{itemize}
  \item Functions definitions are written as equations
  \item double x = x + x

quadruple x = double (double x)
  \item compute the length of a list

-- length, if empty, is zero

length [] = 0

-- length if not empty is one plus the length of its tail

length)x:xs) = 1 + length xs
\end{itemize}


\textbf{Type Polymorphism}


\begin{itemize}
\item What is the type of length?

$>$ length [1,2,3]

3

$>$ length [’a’,’b’,’c’,’d’]

4

$>$ length [[],[1,2],[3,2,1],[],[6,7,8]]

5

\item length works for lists of elements of arbitrary type

length :: [a] -> Int

Here, 'a' denotes a type variable, so the above reads as "length takes a list of arbitray type a and returns an Int"

\end{itemize}


\textbf{Laziness}

\begin{itemize}
\item from n = n : (from (n+1))

This recursive definition generates an infinite list of ascending numbers

\item take n list - return first n elements of list

\item take 10 (from 1)

Haskell is a lazy language, so values are evaluated only when needed
\end{itemize}

\textbf{Program Compactness}


\begin{itemize}
\item Sorting the empty lsit gives the empty list

qsort [] = []

qsort (x:xs)

= qsort [y | y <- xs, y < x]

	++ [x]

	++ qsort [z | z <- xs, z >= x]

\item Haskell list comprehensions

[y | y <- xs, y < x] "build list of ys, where y is drawn from xs such that y < x"
\end{itemize}

\textbf{Patterns in Mathematics}

We characterise things in maths by the laws they obey, laws which often look like patterns:

0! = 1

$n! = n *(n -1)!$,         n > 0

len($\langle \rangle$) = 0

len(L1 L2) = len(L1) + len(L2)

$\langle \rangle$ denotes an empty list and L1L2 is a list concatenation


\textbf{Factorial as Patterns}

Maths:

0! = 1
$n! = n * (n -1)!$


Haskell (without patterns):

\includegraphics{fact}


Haskell (with patterns):

factorial 0 = 1
factorial n = n * factorial (n-1)

Formal argument 0 is shorthand saying check the argument to see if it is zero. If so, do the righthand

Formal argument n says take the argument and refer it to the righthand side as n


\textbf{Lists in Haskell}
There is a standard approach to constructing lists:
\begin{itemize}
\item Empty list using []
\item Given a term x and a list xs we can construct a lsit consisting of x followed by xs as follows: x:xs
\item So the lsit 1,2,3 can be built as 1:2:3:[]. Brackets show how it is built up: 1:(2:(3:[]))
\item We can write [1,2,3] as shorthand for the above list
\item Lists can contain charachters ['H', 'E', 'L', 'L', 'O'] but can be written shorthand as "HELLO".
\item [] and : are list constructors. ":" is pronounced "cons"
\end{itemize}

\textbf{Length with Patterns}

Math:

len($\langle \rangle$) = 0

len(L1L2) = len(L1) + len(L2)
len($\langle _ \rangle$) = 1
len($\langle _ \rangle$L) = 1 + len(L)

Haskell:

mylength [] = 0
mylength (x:xs) = 1 + mylength xs

The key idea in pattern-matching is that the syntax used to build values can also be used to look at a value, determine how it was built and extract out the individual sub-parts if required.


\textbf{Compact "Truth Tables"}

Patterns can be used to give an elegant expression to certain functions, for instances we can define a function over two boolean arguments like this:

myand True True = True
myand \textunderscore       \textunderscore = False

Here, the underscore pattern "\textunderscore" is a wildcard that matches anything


\textbf{Types}

Haskell is strongly typed - every expression/vale has a well-defined type:

 myExpr :: MyType

"The value myExpr has type MyType"

Haskell supports type-inference. We dont have to declare types of functions in advance.


\textbf{Atomic Types}

Some Atomic types builtin to Haskell

\begin{itemize}
\item () - the unit type has only one value, also written as ().
\item Bool - boolean values, of which there are only two: true and false
\item Char - charachter values, representing Unicode charachters
\item Int - fixed-precision integer type
\item Intger - infinite-precision integer ttoe
\item Float - floating point number of precision at least that of IEEE single-precision
\item Double - floating point of precision at least that of IEEE double-precision
\end{itemize}


\textbf{Composite Types}

A type built on top of another type

\begin{itemize}
\item Functions have a type that indicates the type(s) of its input(s) and the type of its output
\item Tuples gather values of other types together in a package
\item Algebraic are data types that allow you to definte types whose valyes can have more than one form

\end{itemize}


\textbf{Tuples}
\begin{itemize}

\item We can create pairs, triples, n-tuples of values: (1,2) or (1, 'a', "Hi!")
\item The type of the pair (42, 'z') where 42 has type Int, is (Int, Char)

e.g. (42 'z') :: (Int,Char)

\item We can use tuples as patterns in function definitions:

sumPair (a,b) = a + b

\end{itemize}


\textbf{Algebraic Data Types}

Haskell lists are an example, with data that can have one of two forms:

Empty [] or non-empty (x:xs)


\textbf{Function Types}

A function type consists of the input tyoe, follow by a right-arrow and then the output type

myFun :: MyInputTyoe -> MyOutputType

like f: A -> B in maths

sumInt :: [Integer] -> Integer adds up a lsit of integers

Consider a rounding function that converts a floating point number to a fixed-width integer:

round :: Double -> Int


\textbf{Inferring Function Types}

FunDef - Given a function declaration like f x = e, if e has type b and (the use of) x (in e) has type a, then f must have type a -> b.

FunUse - Given a function applicayion f v, if f has type a -> b, then v must have type a, and f v will have type b.

\end{document}