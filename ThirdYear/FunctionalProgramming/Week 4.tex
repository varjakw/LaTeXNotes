\documentclass{article}
\usepackage{amsmath}
\usepackage{xcolor}
\usepackage{graphicx}
\graphicspath{ {./images/} }
\title{Functional Programming - Week 4}
\author{Varjak Wolfe, 18325326}
\begin{document}

\textbf{Lambda Abstraction}

This is an old notation for function-values where $f = \lambda x . x + 2$

In Haskell,$\lambda$ becomes \textbackslash and . becomes ->

\textcolor{green}{\textbackslash x -> e}

where x is a variable and e is an expression that mentions x

We can have nested abstractions \textcolor{green}{\textbackslash x -> \textbackslash y -> e} read as "the function taking x as input and returning a function that takes y as input and returns e as a result

\textcolor{green}{sqr = \textbackslash n -> n * n} 

is the eqvuivalent to:

\textcolor{green}{sqr  n = n * n}


\textbf{Factorial}

A simple definition of factorial is:

\textcolor{green}{fac  0 = 1}

\textcolor{green}{fac  n = n * fac  (n-1)}

But what is \textcolor{green}{fac}?

\textcolor{green}{fac = \textbackslash n -> case n of}

\textcolor{green}{0 -> 1}

\textcolor{green}{m -> m * fac  (m-1)}

Here we use a haskell case expression that does pattern matching in a general setting


\textbf{Defining New Types - 3 Ways}

\begin{itemize}
\item Type Synonyms

\textcolor{green}{type Name = String}

Haskell considers String and Name to be exactly the same type
\item Wrapped Types

\textcolor{green}{newType Name = N String}

If s is a value of type String, then \textcolor{green}{N  s} is a value of type Name. Haskell considers String and Name to be different types here.
\item Algebraic Data Types

\textcolor{green}{data Name = Official String String  |  NickName String}

If f,s and n are values of type String, then \textcolor{green}{Official  f  s} and \textcolor{green}{NickName  n} are different values of type Name.

\end{itemize}
\textbf{User-Defined Data Types: enums}

With the data keyword, we can easily defne new enumerated types.

\textcolor{green}{data Day = Monday | Tuesday | Wednesday | Thursday | Friday | Saturday | Sunday}

We can define oeprations on values of this type by pattern matching:

\textcolor{green}{weekend  ::  Day -> Bool}

\textcolor{green}{weekend  Saturday = True}

\textcolor{green}{weekend Sunday  = True}

\textcolor{green}{weekend \textunderscore = False}

The identifiers Monday through Sunday are data constructors, and jsut like the types themselves, must begin with uppercase letters.


\textbf{User-Defined Data Types: Recursive Structurs}

If lists were not builtin, we could define them with the data keyword:

\textcolor{green}{data List = Empty}
\textcolor{green}{|  Node Int List}

Using this def, the list [1,2,3] would be written:

\textcolor{green}{Node  1  (Node  2  (Node  3  Empty))}

Recursive types usually mean recursive functions:

\textcolor{green}{length  ::  List -> Integer}

\textcolor{green}{length  Empty = 0}

\textcolor{green}{length  (Node  \textunderscore  rest) = 1 + (length  rest)}


These lists arent as flexible as the builtins because they are not polymorphic but we can fix hat by using a type variable.

\textcolor{green}{data List t = Empty}
\textcolor{green}{ | Node t (list t)}

No changes to the length function but the type becomes:

\textcolor{green}{length  ::  (List  a) -> Integer}


\textbf{Type Parameters}

The types defined using type, newtype and data can have type parameters themselves:

\textcolor{green}{type TwoList  t = ([t],[t])}

The type "list-of-a" (\textcolor{green}{[a]}) can be considered a parameterized type \textcolor{green}{ [ ]  a}


\textbf{Whats In a Name?}

\textcolor{green}{data MyType = AToken | ANum Int | AList [Int]}

\begin{itemize}
\item the name \textcolor{green}{MyType} after the data keyword is the type name
\item the names \textcolor{green}{AToken, ANum, AList} are data-constructor names
\item type names and data constructor names are in different namespcaes so they can overlap e.g.

\textcolor{green}{ data Thing = Thing String | Thing Int}

\item The same principle applies to to newtypes:

\textcolor{green}{newtype Nat = Nat Int}

\item We call these algebraic data types
\end{itemize}


\textbf{Multiply-Parameterised Data Types}

Here is a useful data type:

\textcolor{green}{data Pair a b = Pair a b}

\textcolor{green}{divmod :: Integer -> Integer -> (Pair Integer Integer)}

\textcolor{green}{divmod x y = Pair (x / y) (x 'mod' y)}

\includegraphics{prelude}

\textbf{Example: failure}

We can write functions such as \textcolor{green}{head} so that they fail outright:

\textcolor{green}{head (x:xs) = x} No [ ] pattern, so runtime fail for head [ ]

Define type \textcolor{green}{Maybe} to represent an optional value of type a.

\textcolor{green}{data Maybe a = Nothing | Just a}

Now we can handle failure for \textcolor{green}{head} in a more manageable way:

\textcolor{green}{mhead :: [a] -> Maybe a}

\textcolor{green}{mhead [ ] = Nothing}

\textcolor{green}{mhead (x:Xs) = Just x}

\textbf{Maybe}

Maybe is a good way to deal with errors or exceptional cases. The \textcolor{green}{Maybe} type encapsulates an optional value. A value of type \textcolor{green}{Maybe a} either contains a value of type a, represented as \textcolor{green}{Just a} or it is empty, represented as \textcolor{green}{Nothing}.

\end{document}

