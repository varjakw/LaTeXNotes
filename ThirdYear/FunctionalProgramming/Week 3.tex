\documentclass{article}
\usepackage{amsmath}
\usepackage{xcolor}
\usepackage{graphicx}
\graphicspath{ {./images/} }
\title{Functional Programming - Week 3}
\author{Varjak Wolfe, 18325326}
\begin{document}

\textbf{Prelude Lists}

See Week 2 slides for type signature etc for prelude list functions.


\textbf{Haskell Syntax}

Case sensitive; these are all differenet names:

ab, aB, Ab, AB

\begin{itemize}
\item Variable Identifiers (varid) start with lowercase e.g. myNAME
\item Constructor Identifiers (conid) start with uppercase letters e.g. Tree
\item Variable Operator (varsym) start with any symbol and continue with symbols and the colon e.g. | : |
\item Constructor Operators (consym) start with a colon and continue with symbols and the colon e.g. : + :
\end{itemize}


\textbf{Operators}

\begin{itemize}
\item Some operators are left associative such as +, -, *, / e.g. \textcolor{green}{a+b+c} parses as \textcolor{green}{(a+b)+c}
\item Some oeprators are right associative such as \&\&, || e.g. \textcolor{green}{a:b:c:[ ]} parses \textcolor{green}{as a:(b:(c:[ ]))}
\item VOther operators are non-associative such as ==./=,<,>,<=,>= e.g.\textcolor{green}{a<= b <= c} is illegal but \textcolor{green}{(a <= b) \&\& (b <= c)} is ok
\end{itemize}


\textbf{Functions}

\begin{itemize}
\item Function application is denoted by juxtaposition and is left-associative
\item \textcolor{green}{f  x  y  z} parses as \textcolor{green}{((f  x)  y)  z}
\item if we want f applied to both x, and to the result of the application of g to y, we must write \textcolor{green}{f  x  (g  y)}
\item In types, the function arrow is right associative \textcolor{green}{Int -> Char -> Bool} passes as \textcolor{green}{Int -> (Char -> Bool)}
\item The type of a function whose first argument is itself a function must be written as \textcolor{green}{(a -> b) -> c}
\end{itemize}


\textbf{Variable Declaration}

\begin{itemize}
\item This can be either a function or a pattern
\item A declaration can also have a lhs that is a single pattern: \textcolor{green}{patn = expr}
\end{itemize}

\end{document}

