\documentclass{article}
\usepackage{amsmath}
\title{Functional Programming $\lambda$ -Calculus - Week 8}
\author{Varjak Wolfe, 18325326}

\begin{document}
\textbf{Syntax}

Infinite set \textbf{Vars}, of variables:

$u, v, x, y, z, ..., x_1, x_2,...$

Well-formed $\lambda$ -calculus expressions LExpr is the smallest set of strings matching the following syntax:

	$M,N,... element of LExpr ::= v$
				$	|	(\lambda x • M)$
				$	|	(M N)$

where x is a random variable and M is an arbitray lambda expression.

a $\lambda$ -Calculus expression is either a variable (v), an abstraction of a variable from an expression $((\lambda x • M))$ or an applciation of one expression to another ((M N)
)

\textbf{Free/Bound Variables}

A variable occurenc is free in an expression if it is not mentioend in an enclosing abtraction.
	x 	($\lambda$ y• (yz))
The z is free because it is not mentioned in the abstraction (which is $\lambda$ y)
The y is bound

A variable can be both free and bound in the same expression

(x($\lambda$ x•(xy))

The first x is a global free variable outside the abstraction
The second x is in (xy) is bound


\textbf{$\lambda$ -Calculus Moves}

$\alpha$ -renaming M -> M'
$\beta$ -reduction 
$\backslash n$ //missed
//here

\textbf{$\alpha$ -Renaming}
($\lambda$ x • ($\lambda$ y •(x y))) -> ($\lambda$ u • $\lambda$ v • (u v)))
($\lambda$ x • (x y)) -/> ($\lambda$ y • (y y))
		formerly free y has been "captured"

Leaves the meaning of a term unchanged. Same as changing the name of a local variable in a program. 

\textbf{Substitution}

Subsituting an expression N for all free occurences of X, in another expression M, written M[N/x]

(x ($\lambda$ y • (z y))) [ ($\lambda$ u • u) / z] -> (x($\lambda$y •(($\lambda$ u • u)y)))

Y is not free wnyehere so substituing in for that would do nothing.

\textbf{Careful Substitution}

When doing general substitution M[N/x], we need to avoid variable capture of free variable sin N by bindings in M:

(x($\lambda$ y•(z y ))) -/> (x ($\lambda$ y • ((y x) y)))

If N has free variables which are oing to be inside an ibastraction on those variables in M, then we need to $\alpha$ - rename the abstractions to something else first and then substitute.
//here
The Golden Rule: //here

\textbf{$\beta$ -Reduction}

($\lambda$ x•M) N -> M[N/x]

We define an eexpression of the form 
($\lambda$ x•M) N as a $\beta$ -redex (reducible expression)


\textbf{How this applies to functions}

$f(x) = 2x + 1
f(42) = substitute 42 for x in the r.h.s
(2x+1)[42/x]
2x42+1$

This is basically $\beta$ -reduction

\textbf{Normal Form}
An expression with no redexes. 

Aim: reduce an expression to its normal form (if it exists)
i.e. play the game until you cant make anymore moves

   ((($\lambda$ x • ($\lambda$ y • (y x))) u) v)
-> (($\lambda$ y • (y u)) v)
-> (v u)	A normal form

    (($\lambda$ x • (x x)) u)
-> ( u u)	A normal form

Not all expressions have normal form.

\textbf{Normal Form(III)}
//here innermost and outermost

\textbf{n reduction (eta?))}

\textbf{$\lambda$ - Calculus and Computability}

We can use it to encode booleans, numebrs and finctions
$\lambda$ - Calculus is Turing-complete

\textbf{How Important is Reduction Order?}


The Church-Rosser theorem states:
If we can reduce M to $N_1$, by one set of redex choices, and to $N_2$ by another, then there always exists a third value R, to which both $N_1$ and $N_2$ can be reduced.

This third value R may be different but could also be one of $N_1$ or $N_2$

\textbf{Normal Forms are Unique}

We reduce M to  $N_1$, where  $N_1$ is a normal form using some reduction sequence. 
We reduce M to  $N_2$, where  $N_2$ is also a normal form using some reduction sequence. 

By the Diamond Lemma, there exists an R to which both $N_1$ and $N_2$ can be reduced.
But $N_1$ and $N_2$ are normal forms so cant be reduced further. Therefore $N_1$ = R = $N_2$



\textbf{Reduction Order}

If we have a chocie of redexes, which should we reduce first?

Always the leftmost-outermost one (Normal Order Reduction)

Haskell uses the Graph Execution model (a variant of NOR) which gives rise to Lazy Evaluation.



\textbf{Lambda Abstraction in Haskell}

$\lambda$ becomes "\"
(x -> 2 * x + 1) 42


Haskell's default behaviour is to reduce an expression to a notion of a partial/incomplete form. Weak Head Normal Form (WHNF)

A Haskell expression is WHNF when its top-level is either:

A data constructor applied to some or all of its values

or

A function that has not been applied to all of its arguments

Largely implemneted by the way pattern-matching is

\textbf{Strictness}

Strict functions always evaluate all their arguments to Normal Form and then produce their own results as a Normal Form.


\textbf{Example: isOddl}
//here



\end{document}