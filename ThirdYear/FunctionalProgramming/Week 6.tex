\documentclass{article}
\usepackage{amsmath}
\usepackage{xcolor}
\usepackage{graphicx}
\graphicspath{ {./images/} }
\title{Functional Programming - Week 6}
\author{Varjak Wolfe, 18325326}
\begin{document}

\textbf{Some operators are "nice"}

Some operators have nice properties like having unit values e.g. $0+a = a = a+0$

We can code a simplifier for these as follows:

\textcolor{green}{uopSimp cons u (Val v) e | v == u = e}

\textcolor{green}{uopSimp cons u e (Val v) | v == u = e}

\textcolor{green}{uopSimp cons u e1 e2 = cons e1 e2}

We can deduce that cons is:

\textcolor{green}{cons :: Expr -> Expr -> Expr}

So we use Add and Mul directly:

\textcolor{green}{simp (Add e1 e2) = uopSimp Add 0.0 e1 e2}

\textcolor{green}{simp (Mul e1 e2) = uopSimp Mul 1.0 e1 e2}


The data constructors of \textcolor{green}{Expr} are in fact functions, whose types are as follows:

\includegraphics{data}

So, cons has to have type \textcolor{green}{Expr -> Expr -> Expr}, which is why Add and Mul are suitable arguments to pass into \textcolor{green}{uopSimp}


Given declaration:

\textcolor{green}{data MyType = ... | MyCons T1 T2 ... Tn | ....}

then data constructor MyCons is a function of type:

\textcolor{green}{MyCons :: T1 -> T2 -> ... -> Tn -> MyType}

Partial applications of MyCons are also valid

\textcolor{green}{(MyCons x1 x2) :: T3 -> ... -> Tn -> MyType}

Data constructors are the only functions that can occur in patterns.


\textbf{Abstractions}

A lot of the higher-order functions in the Prelude are examples of abstraction of common programming shapes encountered in functional programs (e.g. map and folds).

\textbf{Turning Common Shapes Into Functions}

Remember these?

\textcolor{green}{sum [ ] = 0}
\textcolor{green}{sum (n:ns) = n + sum ns}

\textcolor{green}{length [ ] = 0}
\textcolor{green}{length (\textunderscore:xs) = 1 + length xs}

\textcolor{green}{prod [ ] = 0}
\textcolor{green}{prod (n:ns) = n * prod ns}

They have a common pattern which is typically referred to as folding. Can we abstract this?

\textbf{Commonalities}

\begin{itemize}
\item They all have the empty list as a base case
\item They all have a non-empty list as the recursive case
\item The base case returns a fixed "unit" value, which we will call \textcolor{red}{u}

\textcolor{red}{$<$abs-fold$>$ [ ] = u}

\textcolor{red}{$<$abs-fold$>$ (a:as) = ..... $<$abs-fold$>$ as}

\end{itemize}
\end{document}

