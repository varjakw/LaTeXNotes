\documentclass{article}

\title{Computational Maths - Chapter 2}
\author{Varjak Wolfe}

\begin{document}
\maketitle

This follows Chapter 2 of the textbook.
-Intermediate value theorem
-Mean Value Theorem


\textbf{Problem 2.2}

Apply the intermediate value theorem to show that the function  \(f(x) = cosx -x^2\) has a root in the 
interval $[O, \pi/2]$. 

\textbf{Intermediate Value Theorem}

The intermediate value theorem is a useful theorem about the behavior of a function in a closed interval. 
It states that if f(x) is continuous on the closed interval [a, b] and M is any number between f(a) 
and f(b), then there exists at least one number c in [a, b] such that 
f(c) = M (Fig. 2-2). 


The closed interval $[O, \pi/2]$. includes $0$ and $\pi/2$.
The \textbf{Intermediate value theorem} states that there is a c between f(0) and $f(\pi/2)$ such that $f(c) = M$.

If we want to find a root, then M is going to be 0.

From \(f(x) = cosx -x^2\), we find that $f(0) = 1$ and $f(\pi/2) = -\pi^2/4$.

Therefore, we see that when M = 0, \(-\pi^2/4 < M < 1\) i.e. $0$ is between the two values of $f(0)$ and $f(/2)$. 

By the Intermediate Value Theorem, we have at least one root in the closed interval $[O, \pi/2]$


\textbf{Problem 2.8}

As a highway patrol officer, you are participating in a speed trap. A car passes your patrol car which 
you clock at 5 5 mph. One and a half minutes later, your partner in another patrol car situated two miles 
away from you, clocks the same car at 50 mph. Using the mean value theorem for derivatives (Eq. (2.4)), 
show that the car must have exceeded the speed limit of 55 mph at some point during the one and a half 
minutes it traveled between the two patrol cars. 

\textbf{Mean Value Theorem for Derivatives}

The mean value theorem is very useful in numerical analysis when finding bounds for the order of magnitude of numerical error for different 
methods. Formally, it states that if f(x) is a continuous function on the 
closed interval [a, b] and differentiable on the open interval (a, b) , then 
there exists a number c within the interval, c E (a, b) , such that:

\[f (c) = \frac{dx}{dy} = \frac{f(b) - f(a)}{b-a}\]



Using the MVT, consider the rate of change of position with respect to time, x(t). 

At $t=0, x = 0 miles$
At $t = 1.5mins = 0.025hrs, x = 2 miles$.

If x(t) is continuous on the closed interval [0,2] and differentiable on the open interbal (0,2), then the \textbf{mean value theorem for derivatives} states there exists some $t = c$ between $t = 0$ and $t = 1.5$ such that




\[\frac{dx}{dy} = \frac{x(0.025)-x(0)}{0.025-0} = \frac{2}{0.025}mph = 80mph  \]

Clearly, the car must have been speeding at some point between the two patrol cars at t = 0 and at t = 1.5mins.

\textbf{Problem 2.22}
Write the Taylor's series expansion of the function $f(x) = sin(ax) about x = 0, where a=/= 0$ is a 
known constant

\textbf{Taylor Series}
Taylor series expansion of a function is a way to find the value of a 
function near a known point, that is, a point where the value of the function is known. The function is represented by a sum of terms. In some cases (if the function is a polynomial), the Taylor series can give the exact value of the function. In most cases, however, a sum of an infinite number of terms is required for the exact value. If only a few terms are used, the value of the function that is obtained from the Taylor series is an approximation.

\underline{Taylor Series for Function of One Variable}

Expansion of $f(x)$ about a point $x = a$ is given by

\[f (x) = f(a) + f'(a)(x-a) + \frac{f ''(a)}{2!}(x-a)^2 + \frac{f ^{(3)}(a)}{3!}(x-a)^3 + ...+\frac{f ^{(n)}(a)}{n!}(x-a)^n \]


\end{document}