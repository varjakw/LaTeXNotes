
\documentclass{article}
\usepackage{amsmath}
\title{Computational Maths - Chapter 5}
\author{Varjak Wolfe}

\begin{document}
\maketitle

This follows Chapter 5 of the textbook.
-Eigenvalues

-Eigenvectors

-Power Method



\textbf{Background}

For an N x N matrix [a], the number $\lambda$ is an eigenvalue of the matrix if:
\[[a][u] =  \lambda[u] \]

Vector [u] is a column vector with n elements called the eigenvector, associated with the eigenvalue  $\lambda$.

A more general form of the operation is
\[L[u] =  \lambda[u] \]
where [a][u] is a mathematical operation and can be thought of as the matrix [a] operating on the operand [u]. L is an operator that can represent multiplication by a matrix, differentiation, integration etc. u is a vector or function and $\lambda$ is a scalar constant. 

For example, if L represents second differentiation with respect to x, y is a function of x and k is a constant, then it can be written as:

\[ \frac{d^2y}{dx^2} = k^2y\]

The previous equation \textbf{\(L[u] =  \lambda[u\)} is a general statement of an eigenvalue problem where $\lambda$ is called the eigenvalue associated with operator L and u is the eigenvector/eigenfunction corresponding to $\lambda$ and L.

\textbf{The Characteristic Equation}

\[[a-\lambda I][u] = 0\] where [I] is the indentity matrix with the same dimensions as [a]. If the matrix \([a-\lambda I]\) has an inverse, then multiplying both sides of the equation by the inverse will give the trivial solution $[u] = 0$. If it does not have an inverse, a nontrivial solution for [u] is possible.

Another way of stating this criterion based on Cramer's Rule from chapter 2: the matrix $[a-\lambda I]$ is singular if its determinant is zero:
\[ det[a-\lambda I] - 0\]

For a given matrix [a], it yields a polynomial equaron for $\lambda$ whose roots are the eigenvalues. Once the eigenvalues are known, we can find the eigenvectors by substituing the eigenvalues one at a time into \([a][u] =  \lambda[u] \)


\textbf{The Basic Power Method}

This is an iterative procedure for determining the largest real eigenvalue and the corresponding eigenvector of a matrix. Consider $N x N$ matrix [a] that has n distinct real eigenvalues \(\lambda 1, \lambda 2, ..., \lambda n\) and n associated eigenvectors \([u]1,[u]2, ..., [u]n\). The eigenvalues are numbered from largest to smallest such that  \[\lambda 1 < \lambda 2 < \lambda n\]

1. Start with a column eigenvector $[x]i$ of length n. The vector can be any nonzero vector.

2. Multiply the vector $[x]i$ by the matrix [a]. The result is a column vector$[x]_{i+1}, [x]_{i+1} = [a][x]_i$

3.  Normalize the resulting vector $[x]_{i+1}$. This is done by factoring out the largest element in the vector. The result of this operation is a multiplicative factor (scalar) times a normalized vector. The normalized vector has the value 1 for the element that used to be the largest, while the absolute values of the rest of the elements are less than 1. 

4. Assign the normalized vector (without the multiplicative factor) to $[x]_i$ and go back to step 1. 

The iterations continue until the difference vetween vector $[x]_i$ and the normalized vector $[x]_{i+1}$ is less than some specified tolerance.

The last multiplicatiive factor is the largest eigenvalue, and the normalized vector is the associated eigenvector.

\textbf{Example 5-2: Using The Power Method to Determine the Largest Eigenvalue of a Matrix}

\begin{equation*}
\begin{pmatrix}
4 & 1 & -2 \\
-2 & 8 & 1 \\
2 & 4 & -4 \\
\end{pmatrix}
\end{equation*}

Use the Power Method and start with vector $x = [1, 1, 1]^T$

Starting with $i = 1, x_1 = [1, 1, 1]^T$. With the power method, the vector $[x]_2$ is first calculated by $[x]_2 = [a][x]_1$ (step 2) and is then normalized (step 3):

\begin{equation*}
[x]_2 = [a][x]_1 =
\begin{bmatrix}
4 & 2 & -2 \\
-2 & 8 & 1 \\
2 & 4 & -4 \\
\end{bmatrix}
\begin{bmatrix}
1 \\
1 \\
1 \\
\end{bmatrix}
=
\begin{bmatrix}
4 \\
7\\
2 \\
\end{bmatrix}
= 7
\begin{bmatrix}
0.5714 \\
1\\
0.2857 \\
\end{bmatrix}
\end{equation*}


For $i = 2$, the normalized vector $[x]_2$ (without the multiplicative factor) is multiplied by [a]. This results in $[x]_3$ which is then normalized:

\begin{equation*}
[x]_3 = [a][x]_2 =
\begin{bmatrix}
4 & 2 & -2 \\
-2 & 8 & 1 \\
2 & 4 & -4 \\
\end{bmatrix}
\begin{bmatrix}
0.5714 \\
1 \\
0.2857 \\
\end{bmatrix}
=
\begin{bmatrix}
3.7143 \\
7.1429 \\
4 \\
\end{bmatrix}
= 7.1429
\begin{bmatrix}
0.52 \\
1\\
0.56 \\
\end{bmatrix}
\end{equation*}

$i = 3$

\begin{equation*}
[x]_4 = [a][x]_3
\begin{bmatrix}
4 & 2 & -2 \\
-2 & 8 & 1 \\
2 & 4 & -4 \\
\end{bmatrix}
\begin{bmatrix}
0.52\\
1 \\
0.56 \\
\end{bmatrix}
=
\begin{bmatrix}
2.96 \\
7.52 \\
2.8 \\
\end{bmatrix}
= 7.52
\begin{bmatrix}
0.3936 \\
1\\
0.3723 \\
\end{bmatrix}
\end{equation*}

And so on.

As of 8 iterations, the results are:

$i = 8$

\begin{equation*}
[x]_9 = [a][x]_8
\begin{bmatrix}
4 & 2 & -2 \\
-2 & 8 & 1 \\
2 & 4 & -4 \\
\end{bmatrix}
\begin{bmatrix}
0.3272\\
1 \\
0.3946 \\
\end{bmatrix}
=
\begin{bmatrix}
2.5197 \\
7.7401 \\
3.0760 \\
\end{bmatrix}
= 7.7401
\begin{bmatrix}
0.3255 \\
1\\
0.3974 \\
\end{bmatrix}
\end{equation*}

The results show that the differences between the vector $[x_i] and the normalized vector [x_{i+1}] are getting smaller. The value of the multiplicative factor (7.7401) is an estimate of the largest eigenvalue.


\textbf{The Inverse Power Method}
\textbf{The Shifted Power Method}
\textbf{The QR Factorization and Iteration Method}






\end{document}