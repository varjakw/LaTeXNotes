\documentclass{article}
\usepackage{amsmath}
\usepackage{textcomp}
\title{Computational Maths - Assignment 2}
\author{Varjak Wolfe, 18325326}

\begin{document}
\maketitle

Assignment 2 of Computational Mathematics.


\textbf{Question 1}

Using MATLAB with the given script gave me an answer of logical 1.

$X=(1$\textasciitilde$=0)|(2>2)\&(7<4);$

Here, the first bit is asking if 1 is not equal to 0 (\textasciitilde=)., which it is, so answer is true i.e. 1

The symbol | is logical OR

The symbol \& is logical AND

$(2>2)\&(7<4)$ is equivalent to (0 \& 0), because both are false and thus 0.

So, X = 1 OR 0

X = 1

Answer is A


\textbf{Question 2}

Upon running the given script, the result were:

   -7.7417

   -0.2583

The script:

p=[1 8 2]; 

r=roots(p);

where p is polynomial represented by a row vector with elements 1,8,2. 
[1 8 2] represents $x^2 + 8x + 2$. Using the quadratic formula, we get results very close to matlab's.

Answer is E


\textbf{Question 3}

a=12/1*15/1;  	= 180

b= a/a*a; 		= 180

c=tand(30)+1/3; 	=  0.9107

d=1+c; 		=   1.9107

e = a-b*c+d;	= 17.9876

Answer is E

\textbf{Question 4}


Use the Power Method and start with vector $x = [1, -0.8, 0.9]^T$. Perform 7 iterations.

\underline{Starting with $i = 1$},$ x_1 = [1, -0.8, 0.9]^T$.
\newline
With the power method, the vector $[x]_2$ is first calculated by $[x]_2 = [a][x]_1$ (step 2) and is then normalized (step 3):

\begin{equation*}
[x]_2 = [a][x]_1 =
\begin{bmatrix}
-7 & 13 & -16 \\
13 & -10 & 13 \\
-16 & 13 & -7 \\
\end{bmatrix}
\begin{bmatrix}
1 \\
-0.8 \\
0.9 \\
\end{bmatrix}
=
\begin{bmatrix}
-31.8 \\
32.7\\
-32.7 \\
\end{bmatrix}
= 32.7
\begin{bmatrix}
 -0.9725\\
1\\
-1\\
\end{bmatrix}
\end{equation*}

This is done by taking 32.7 out and dividing each element of the matrix 

$
\begin{bmatrix}
-31.8 \\
32.7\\
-32.7 \\
\end{bmatrix}
$
by 32.7, giving the result
$
= 32.7
\begin{bmatrix}
 -0.9725\\
1\\
-1\\
\end{bmatrix}
$
\newline

\underline{Now with $i = 2$}


\begin{equation*}
[x]_3 = [a][x]_2 =
\begin{bmatrix}
-7 & 13 & -16 \\
13 & -10 & 13 \\
-16 & 13 & -7 \\
\end{bmatrix}
\begin{bmatrix}
 -0.9725\\
1\\
-1\\
\end{bmatrix}
=
\begin{bmatrix}
35.8075\\
-35.6425\\
 35.56\\
\end{bmatrix}
= -35.6425
\begin{bmatrix}
-1.005\\
1\\
-0.9977\\
\end{bmatrix}
\end{equation*}

\underline{Now with $i = 3$}


\begin{equation*}
[x]_4 = [a][x]_3 =
\begin{bmatrix}
-7 & 13 & -16 \\
13 & -10 & 13 \\
-16 & 13 & -7 \\
\end{bmatrix}
\begin{bmatrix}
-1.005\\
1\\
-0.9977\\
\end{bmatrix}
=
\begin{bmatrix}
35.9982\\
-36.0351\\
36.0639\\
\end{bmatrix}
=-36.0351
\begin{bmatrix}
-0.999\\
1\\
-1\\
\end{bmatrix}
\end{equation*}

\underline{Now with $i = 4$}


\begin{equation*}
[x]_5 = [a][x]_4 =
\begin{bmatrix}
-7 & 13 & -16 \\
13 & -10 & 13 \\
-16 & 13 & -7 \\
\end{bmatrix}
\begin{bmatrix}
-0.999\\
1\\
-1\\
\end{bmatrix}
=
\begin{bmatrix}
35.993\\
-35.987\\
35.984\\
\end{bmatrix}
=-35.987
\begin{bmatrix}
-1\\
1\\
-0.999\\
\end{bmatrix}
\end{equation*}

\underline{Now with $i = 5$}


\begin{equation*}
[x]_6 = [a][x]_5 =
\begin{bmatrix}
-7 & 13 & -16 \\
13 & -10 & 13 \\
-16 & 13 & -7 \\
\end{bmatrix}
\begin{bmatrix}
-1\\
1\\
-0.999\\
\end{bmatrix}
=
\begin{bmatrix}
35.984\\
-35.987\\
35.993\\
\end{bmatrix}
=-35.987
\begin{bmatrix}
-0.999\\
1\\
-1//
\end{bmatrix}
\end{equation*}

\underline{Now with $i = 6$}


\begin{equation*}
[x]_7 = [a][x]_6 =
\begin{bmatrix}
-7 & 13 & -16 \\
13 & -10 & 13 \\
-16 & 13 & -7 \\
\end{bmatrix}
\begin{bmatrix}
-0.999\\
1\\
-1//
\end{bmatrix}
=
\begin{bmatrix}
35.993\\
-35.987\\
35.984\\
\end{bmatrix}
=-35.987
\begin{bmatrix}
-1\\
1\\
-0.999\\
\end{bmatrix}
\end{equation*}

\underline{Now with $i = 7$}


\begin{equation*}
[x]_8 = [a][x]_7 =
\begin{bmatrix}
-7 & 13 & -16 \\
13 & -10 & 13 \\
-16 & 13 & -7 \\
\end{bmatrix}
\begin{bmatrix}
-1\\
1\\
-0.999\\
\end{bmatrix}
=
\begin{bmatrix}
35.984\\
-35.987\\
35.993\\
\end{bmatrix}
=-35.987
\begin{bmatrix}
-0.999\\
1\\
-1//
\end{bmatrix}
\end{equation*}

\underline{Now with $i = 8$}


\begin{equation*}
[x]_9 = [a][x]_8 =
\begin{bmatrix}
-7 & 13 & -16 \\
13 & -10 & 13 \\
-16 & 13 & -7 \\
\end{bmatrix}
\begin{bmatrix}
-0.999\\
1\\
-1//
\end{bmatrix}
=
\begin{bmatrix}
35.993\\
-35.987\\
35.984\\
\end{bmatrix}
=-35.987
\begin{bmatrix}
-1\\
1\\
-0.999\\
\end{bmatrix}
\end{equation*}

After 8 iterations, I have gotten -36, [-1, 1, -1], leading me to answer E. None of the above.
It is close to answer C so perhaps some of my signs are wrong but I will answer E.


\textbf{Question 5}


\begin{center}
\begin{tabular}{ |c|c|c|c|c|c|c|c|c| } 
 \hline
$T_k$ & 0 & 1 & 2 & 3 & 4 & 5 & 6 & 7\\
\hline
$S_k$)& 1.15 & 2.32 & 3.32 & 4.53 & 5.65 & 6.97 & 8.02 & 9.23\\
 \hline
\end{tabular}
\end{center}

Testing the linear relationship between T and S


$S = aT + b$


$S_x = \sum^{8}_{i=1}X_i = 28$


$S_y = \sum^{8}_{i=1}Y_i = 41.19$


$S_{xx} = \sum^{8}_{i=1}X_i^2 = 140$


$S_{xy} = \sum^{8}_{i=1}X_iY_i = 0 + 2.32 + 6.64 + 13.59 + 22.6 + 34.85 + 48.12 + 64.61 = 192.73$


\[a_1 = \frac{nS_{xy} - S_{x}S_y}{nS_{xx}-(S_x)^2} = \frac{(8 * 192.73) - (28*41.19)}{(8 * 140)-(28)^2} =  -1.138\]

\[a_0 = \frac{S_{xx}S_y - S_{xy}S_x}{nS_{xx}-(S_x)^2} = \frac{(140 * 41.19) - (192.73 * 28)}{(8 * 140-(28)^2} = 1.1016 \]

The equation of best fit is $ y = a_1x + a_0$

$S = -1.138T + 1.1016$

The answer is E


\textbf{Question 6}

(1, 5.12), (3, 3), (6, 2.48), (9, 2.34), (15, 2.18)

\[f(x) = \alpha e^{\beta x}\]

For $y = be^{mx}, b = \alpha, m = \beta$ 

$ln(y) = mx + ln(b)$

In $ Y = a_1X + a_0$,
\begin{itemize}
\item $ Y = ln(y)$
\item $X = x$
\item $a_1 = m$
\item $a_0 = ln(b)$
\end{itemize}

The values for the linear least squares regression are $x_i$ and $ln(yi)$


$S_x = \sum^{5}_{i=1}X_i = 1 + 3 + 6 + 9 + 15 = 34$


$S_y = \sum^{5}_{i=1}ln(Y_i) = ln(5.12) + ln(3) + ln(2.48) + ln(2.34) + ln(2.18) = 5.269 $


$S_{xx} = \sum^{5}_{i=1}X_i^2 = 352$


$S_{xy} = \sum^{5}_{i=1}X_iln(Y_i) = 1.633 + 3.296 + 5.4496 + 7.651 + 11.6899 = 29.7195 $


\[a_0 = \frac{S_{xx}S_y - S_{xy}S_x}{nS_{xx}-(S_x)^2} = 1.3977  \]

\[a_1 = \frac{nS_{xy} - S_{x}S_y}{nS_{xx}-(S_x)^2} = - 0.050577 \]



Approximately 1.3980, -0.050601

Answer is A


\textbf{Question 7}

(1, 5.12), (3, 3), (6, 2.48), (9, 2.34), (15, 2.18)

\[ln(y) = mln(x) + ln(b)\]

\[Y = a_1X + a_0\]

\[g(x) = \alpha + \frac{\beta}{x}\]

Take g(x) as:

\[y = \beta(\frac{1}{x}) + a\]

Convert it to normal form by using X = 1/x
to get $ Y = \beta X + \alpha$

Treat $X_i$'s as $\frac{1}{x_i}$


$ln(y) = mx + ln(b)$

$ Y = a_1X + a_0$,


The values for the linear least squares regression are $1/x_i$ and $yi$


$S_x = \sum^{5}_{i=1}1/X_i = (1/1 + 1/3 + 1/6 + 1/9 + 1/15) = 1.6778$


$S_y = \sum^{5}_{i=1}Y_i = 5.12 + 3 + 2.48 + 2.34 + 2.18 = 15.12$


$S_{xx} = \sum^{5}_{i=1}(\frac{1}{X_i})^2 =1 + 1/9 + 1/36 + 1/81 + 1/225 = 1.155679 $


$S_{xy} = \sum^{5}_{i=1}\frac{1}{X_i}Y_i = 5.12 + 1 + 0.413 + 0.26 + 0.1453 = 6.9383$


\[a_0 = \frac{S_{xx}S_y - S_{xy}S_x}{nS_{xx}-(S_x)^2} = 1.9683  \]

\[a_1 = \frac{nS_{xy} - S_{x}S_y}{nS_{xx}-(S_x)^2} = 3.1461 \]



Approximately 1.9681, 3.1468

Answer is A


\textbf{Question 8}

Newton's Interpolating Polynomial

\[f(x) = log_4(cos(x))\]

$x_1 = 0.5$

$ x_2 = 1.0$

$ x_3 = 1.5$

\[f(0.5) = log_4(cos(0.5)) = -0.0942 = y_1 \]

\[f(1.0) = log_4(cos(1.0)) = -0.4444 = y_2 \]

\[f(1.5) = log_4(cos(1.0)) = -1.911 = y_3 \]

Evaluate $P_2(x)$ at $x = 1.3$

\[P_2(x) = a_1 + a_2(1.3-0.5) + a_3(1.3 - 0.5)(1.3 - 1.0)\]

\[P_2(x) = a_1 + a_2(0.8) + a_3(0.8)(0.3)\]

Sub in the two points to get:

\[y_2 = y_1 + a_2(x_2 - x_1)\]
or
\[a_2 = \frac{-0.4444 - (-0.0942)}{1.0-0.5}\]

\[a_2 = -0.7004\]

Subbing the third point:

\[y_3 = y_1 + \frac{y_2-y_1}{x_2-x_1}(x_3-x_1) + a_3(x_3-x_1)(x_3-x_2)\]

\[-1.911 = -0.0942 + \frac{-0.4444-(-0.0942)}{1-0.5}(1.5-0.5) + a_3(1.5-0.5))(1.5-1.0)\]

\[-1.911 = -0.0942 + (-0.7004)(1.5-0.5) + a_3(1.5-0.5))(1.5-1.0)\]

\[-1.911 = -0.7946 + a_3(1.5-0.5))(1.5-1.0)\]

\[-1.911 = -0.7946 + a_3(0.5)\]

\[-1.911  + 0.7946 = a_3(0.5)\]

\[-1.1164 = a_3(0.5)\]

\[-2.2328\]

This is totally wrong lol. I did it in lagrange and got D so im not sure what happened


\textbf{Question 9}

Evaluate the Lagrange interpolating polynomial

\[f(x) = x^3log_2(x)\]

$x_1 = 2$

$ x_2 = 3$

$ x_3 = 7$

To get y values:

\[f(2) =  (x)^3log_2(x) = y_1 = 8\]

\[f(3) =  (x)^3log_2(x) = y_2 = 42.794 \]

\[f(7) =  (x)^3log_2(x) = y_3 = 962.9227 \]

Evaluate $P_2(x)$ at $x = 5$

\[P_2(x) = \frac{(x-x_2)(x-x_3)}{(x_1 - x_2)(x_1 - x_3)}y_1 +  \frac{(x-x_1)(x-x_3)}{(x_2 - x_1)(x_2 - x_3)}y_2 + \frac{(x-x_1)(x-x_2)}{(x_3 - x_1)(x_3 - x_2)}y_3\]


\[P_2(5) = \frac{(5-3)(5-7)}{(2 - 3)(2 - 7)}y_1 +  \frac{(5-2)(5-7)}{(3 - 2)(3 - 7)}y_2 + \frac{(5-2)(5-3)}{(7 - 2)(7 - 3)}y_3\]

\[P_2(5) = (-0.8)y_1 + (1.5)y_2 + (0.3)y_3\]

\[P_2(5) = (-0.8)(8) + (1.5)(42.794) + (0.3)(962.9227)\]

\[P_2(5) = 346.66781\]

Answer is E


\textbf{Question 10}

Evaluate the Lagrange interpolating polynomial, then derive it to get the acceleration

From the textbook: The coefficients a1, and a2 are the same in the first-order and second-order polynomials. This means that if two points are given and a 
first-order Newton's polynomial is fit to pass through those points, and 
then a third point is added, the polynomial can be changed to be of second-order and pass through the three points by only determining the value of one additional coefficient.

This seems relevant to the fact that there are 4 data points but the scientist used a second order equation.

\[f(x) = x^3log_2(x)\]

$x_1 = 9$

$ x_2 = 15$

$ x_3 = 20$

$y_1 = 21$

$ y_2 = 32$

$ y_3 = 48$

1: \[f(x) = a_1 + a_2(x-x_1)\]

\[\frac{f(x) - y_1}{x - x_1} = \frac{y_2 - y_1}{x_2 - x_1}\]

Solve for f(x):

2: \[f(x) = y_1 + \frac{y_2-y_1}{x_2-x_1}(x-x_1)\]

Comparing equations 1 and 2, we get:

$a_1 = y_1$ and $a_2 \frac{y_2-y_1}{x_2-x_1}$

$a_1 = 21$

$a_2 = \frac{32-21}{15-9} = 1.833$

\[f(x) = a_1 + a_2(x-x_1)+a_3(x-x_1)(x-x_2)\]

\[y_3 = y_1 + \frac{y_2-y_1}{x_2-x_1}(x_3 - x_1) + a_3(x_3 - x_1)(x_3 - x_2)\]

Rearranged:

\[a_3 = \frac{\frac{y_3-y_2}{x_3-x_2} - \frac{y_2-y_1}{x_2-x_1}}{(x_3-x_1)}\]

\[a_3 = 0.124 \]

We now have a second order with 3 points. Sub in our x = 18

\[f(x) = a_1 + a_2(x-x_1)+a_3(x-x_1)(x-x_2)\]

Differentiated:

\[f'(x) = a_3(x-x_2) + a_3(x-x_1) + a_2\]

\[f'(18) = 0.124(18-15) + 0.124(18-9) + 1.833\]

\[f'(18) = 3.321\]

Answer is E









\end{document}

