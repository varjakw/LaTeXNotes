
\documentclass{article}
\usepackage{xcolor}
\usepackage{graphicx}
\graphicspath{ {./images/} }
\usepackage{amsmath}
\title{Symbolic Programming -  Chapter 7 - Definite Clause Grammars}
\author{Varjak Wolfe}

\begin{document}
\maketitle

These notes follow the online coursebook Learn Prolog Now.

\textbf{Context-Free Grammars}
DCGs are a special notation for defining grammars.

CFGs are a finite collection of rules which tell us that certain sentences are correct and what their structure is.

A simple context free grammar for a small fragment of English:

s  $->$   np  vp

np  $->$   det  n

vp  $->$   v  np

vp  $->$   v

det  $->$  a

det  $->$  the

n  $->$  woman

n  $->$  man

v  $->$  shoots 


The arrow -> means its a rule. The symbols s, np, vp, det, n, v are non-terminals. In this case, each of them has a meaning from linguistics: s = sentence, np = noun phrase, vp = verb phrase and det = determiner. i.e. each symbol is shorthand for a grammatical category. 

n = noun, v = verb

Finally, we have symbols a, the, woman, man, shoots. These are terminal symbols or words or lexical items.

This grammar contains 9 context free rules. A CFR consists of a single nonterminal, followed by an arrow and a finite sequence made of terminal and/or nonterminals.

Rule 1 tells us a sentence consists of a noun phrase and a verb phrase; and so on.

Is the string "a woman shoots a man" grammatically correct in our CFG?

s $->$  np vp

np $->$  det n

det $->$  a or the

vp $->$ v np or v

A woman shoots a man = det n v det n

s = det n v det n

s = np v np

s = np vp

Therefore it is grammatically correct for our CFG


\textbf{Parse Tree}

The tree that can be used to answer the above question:

\includegraphics{tree}

This is a parse tree; one which represents the syntactic structure of a string. They provide information about the string and its structure.

If we are given a string of words and a grammar, and it turns out we can build a parse tree then we can say that the string is grammatical for this particular grammar.

The language generated by a grammar consists of all the strings that the grammar classifies as grammatical.


\textbf{Recogniser}

A context free recogniser is a program which correctly tells us whether or not a string belongs to the language generated by a CFG. Basically, it classifies strings as either grammatical or ungrammatical.


\textbf{Parser}

A context free parser correctly decides whether a string belongs to the language generated by a context free grammar and it also tells us the structure.


A recogniser says yes or no but a parser also provides a parse tree.



\textbf{CFG Recognition in Prolog}

Using a list to represent a sequence of tokens: [a, woman, shoots, a, man]

The rule s $->$ np vp can be thought as concatenating an np-list with a vp-list resulting in an s-list. We can concatenate using append/3. See recogniser.pl in PrologDCG on GitHub.

The problems with this recogniser:

\begin{itemize}
\item It doesnt use the input string to guide the search
\item Goals such as np(A) and vp(B) are called with uninstantiated variables
\end{itemize}


A more efficient implementation: difference lists



\textbf{Dffierence Lists}

[a,b,c]-[] is the list [a.b.c]

[a,b,c]-[d] is the list [a.b.c]

[a,b,c|T]-T is the list [a.b.c]

X-X is the empty list [ ]

See Recogniser.pl for an implementation

The recogniser using difference lists is a lot more efficient than using append/3. 


\textbf{Definite Clause Grammars}

DCGs have the simplicity of append but the efficiency of difference lists.

They are a nice notation for writing grammars that hides the underlying difference list variables. 

s $-->$ np, vp. 

np $-->$ det, n. 

vp $-->$ v, np. 

vp $-->$ v. 

det $-->$ [the].     
       
det $-->$ [a]. 

n $-->$ [man].    
           
n $-->$ [woman].   
        
v $-->$ [shoots].


A DCG rule such as s $-->$ np, vp. is really a syntactic variant of:

s(A,B):- np(A,C), vp(C.B).


Another DCG example:

s $-->$ s, conj, s.    
       
s --> np, vp. 

np --> det, n.   
           
vp --> v, np.    
         
vp --> v. 
 
det --> [the].  
            
det --> [a]. 

n --> [man].    
            
n --> [woman].            

v --> [shoots]. 

conj --> [and].  
          
conj --> [or].    
        
conj --> [but].

We have added some recursive rules in this case though.


\textbf{DCG Without Left-Recursive Rules}

s $-->$ simple\textunderscore s, conj, s.    
 
s $-->$ simple\textunderscore s. 

simple\textunderscore s $-->$ np, vp. 

np $-->$ det, n.      
         
vp $-->$ v, np.      
        
vp $-->$ v. 
 
det $-->$ [the].  
             
det $-->$ [a]. 

n $-->$ [man].  
            
n $-->$ [woman].   
         
v $-->$ [shoots]. 

conj $-->$ [and].    
      
conj $-->$ [or].    
        
conj $-->$ [but]. 


\textbf{DCGs for Formal Languages}

A formal language is simply a set of strings. We will define the language $a^nb^n$, where the must be the same number of a's as b's.

s $-->$ [ ].

s $-->$ l,s,r.

l $-->$ [a]

r $-->$ [b]



 
\end{document}