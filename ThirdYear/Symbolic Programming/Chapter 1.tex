
\documentclass{article}
\usepackage{xcolor}
\usepackage{amsmath}
\title{Symbolic Programming -  Chapter 1 - Facts, Rules, Queries}
\author{Varjak Wolfe}

\begin{document}
\maketitle

These notes follow the online coursebook Learn Prolog Now.

\textbf{Facts, Queries, Rules}

A collection of facts and rules is called a knowledge base.

Prolog programs are simply knowledge bases which describe some collection of relationships.

We use a prolog program by posing queries; that is, by asking questions about the information stored in the knowledge base.

Facts are used to state things that are unconditionally true of some situation.

e.g: Knowledge Base 1

\textcolor{red}{woman(mia).}

\textcolor{red}{woman(jody).}

\textcolor{red}{woman(yolanda).}

\textcolor{red}{playsAirGuitar(Jody).}

\textcolor{red}{party.}

\textcolor{green}{?- woman(mia).}

Prolog will answer yes because this is one of our stated facts

\textcolor{green}{?- playsAirGuitar(mia).}

Prolog will answer no.

\textcolor{green}{?- party.}

Yes


e.g: Knowledge Base 2

\textcolor{red}{happy(yolanda).}  This is a fact.

\textcolor{red}{happy(mia).}  This is a fact.

\textcolor{red}{listens2Music(yolanda):- happy(yolanda).}   This is a rule. It depends on the condition after :- (if)

\textcolor{red}{playsAirGuitar(yolanda):- listens2Music(yolanda).}

\textcolor{red}{playsAirGuitar(mia):- happy(mia).}


The facts and rules contained in a knowledge base are called clauses. 

Knowledge base 2 contains 2 facts and 3 rules. We also say it has 3 predicates happy, listens2Music and playsAirGuitar.

The happy predicate is defined using a single clause (a fact). The playsAirGuitar predicate is defined using two clauses (one ruke and one fact).

We can also define a rule with two items in the body i.e. two conditions

conjunction:

 playsAirGuitar(vincent):- listens2Music(vincent), happy(vincent). 

Vincent plays air guitar if he listens to music and is happy

disjunction:

 playsAirGuitar(vincent):- listens2Music(vincent); happy(vincent). 

Vincent plays air guitar if he listens to music or is happy


e.g.: Knowledge Base 3

\textcolor{red}{woman(mia).}

\textcolor{red}{woman(jody).}

\textcolor{red}{woman(yolanda).}

 \textcolor{red}{loves(vincent,mia).}

  \textcolor{red}{ loves(marsellus,mia).}

  \textcolor{red}{ loves(pumpkin,honey\_bunny).}

 \textcolor{red}{  loves(honey\_bunny,pumpkin). }

\textcolor{green}{?- woman(X).}

X is a variable. Any word beginning with upper case letter is a variable. This query says: tell me which of the individuals you know about is a woman.


e.g.: Knowledge Base 4

\textcolor{red}{ loves(vincent,mia).}

 \textcolor{red}{loves(marsellus,mia).}

  \textcolor{red}{ loves(pumpkin,honey\_bunny).}

 \textcolor{red}{  loves(honey\_bunny,pumpkin).}
   
 \textcolor{red}{  jealous(X,Y):-  loves(X,Z),  loves(Y,Z). }

Contains 4 facts about the loves relation and one rule.

However, the rule has 3 variables. It is defining a concept of jealousy. X will be jealous of Y if there is some person Z that X loves, and Y loves that same person Z too.

This is a general statement about everyone in the knowledge base, not just specific people.

\textcolor{green}{?- jealous(marsellus, W).}

This asks: can you find a person W such that Marsellus is jealous of them?

The program answers W= vincent.
\end{document}