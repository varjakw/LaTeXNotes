
\documentclass{article}
\usepackage{xcolor}
\usepackage{graphicx}
\graphicspath{ {./images/} }
\usepackage{amsmath}
\title{Symbolic Programming -  Chapter 8 - More DCGs}
\author{Varjak Wolfe}

\begin{document}
\maketitle

These notes follow the online coursebook Learn Prolog Now.

Two important capabilities offered by DCG notation:

- Extra arguments

- Extra tests


\textbf{Extra Arguments}

CDGs allow us to specify extra arguments.

Let us extend our previous grammar from the last chapter to include sentences containing pronounts like she,he.

s $-->$ np, vp. 

np $-->$ det, n. 

vp $-->$ v, np. 

vp $-->$ v. 

det $-->$ [the].     
       
det $-->$ [a]. 

n $-->$ [man].    
           
n $-->$ [woman].   
        
v $-->$ [shoots].

We need to add rules for ponouns and add a rule for saying that noun phrases can be pronouns.

s $-->$ np, vp. 

np $-->$ det, n. 

np $-->$ pro.

vp $-->$ v, np. 

vp $-->$ v. 

det $-->$ [the].     
       
det $-->$ [a]. 

n $-->$ [man].    
           
n $-->$ [woman].   
        
v $-->$ [shoots].

pro $-->$ [he].

pro $-->$  [she].

pro $-->$ [him].

pro $-->$ [her].


Some sample queries:

?- s([she,shoots,him],[]).

yes


?- s([a, woman, shoots, him],[]).

yes

These are grammtoca strings accepted by the DCG. However, there are also examples of ungrammatical strings that it accepts:

?- s([a, woman, shoots, he],[]).

yes


?- s([her, shoots, she],[]).

yes


The DCG ignroes some basic facts about english:

- she and he are subject pronouns while her and him are obejct pronouns. We can do this using extra arguments.

s $-->$ np(subject), vp. 

np(\textunderscore) $-->$ det, n. 

np(X) $-->$ pro(X).

vp $-->$ v, np(object). 

vp $-->$ v. 

det $-->$ [the].     
       
det $-->$ [a]. 

n $-->$ [man].    
           
n $-->$ [woman].   
        
v $-->$ [shoots].

pro(subject) $-->$ [he].

pro(subject) $-->$  [she].

pro(object) $-->$ [him].

pro(object) $-->$ [her].


?- s([she, shoots, he],[])

no




So how does it work?

Recall that the rule 

s $-->$ np,vp

is syntactic sugar for:

s(A.B):-np(A,C), vp(C,B).


So the rule:
 
s $-->$ np(subject), vp.

translates into:

s(A,B):- NP(subject,A,C), vp(C,B).


\textbf{Listing noun phrases:}

?- np(Type, NP, []).

Type =\textunderscore

NP = [the, woman];

Type =\textunderscore

NP = [the, man];

Type =\textunderscore

NP = [a, woman];

Type =\textunderscore

NP = [a, man];

Type =\textunderscore

NP = [he];


\textbf{Building Parse Trees}

\includegraphics{tree}

Lets make our DCG build a parse tree.


\includegraphics{parse}

?- s(T,[he,shoots],[]).

T = s(np(pro(he)),vp(v(shoots)))

yes


\textbf{Beyond Context-Free Languages}

In the previous lecture we showed DCGs as useful for working with context free grammars. However, they can deal with a lot more than that. The extra arguments allow us to cope with any computable language.



\textbf{An example using $a^nb^nc^n / [\epsilon] $ }

This language consists of strings such as abc, aabbcc, aaabbbccc, aaaabbbbcccc and so on.

This is not context-free- but can we still write a DCG to produce these strings.


s(Count) $-->$ as(Count), bs(Count), cs(Count).


as(0) $-->$ [].

as(succ(Count)) $-->$ [a], as(Count).


bs(0) $-->$ [].

bs(succ(Count)) $-->$ [b], bs(Count).


cs(0) $-->$ [].

cs(succ(Count)) $-->$ [c], cs(Count).


We can also call any prolog predicate from the right side of a DCG rule using curly brackets { }


s(Count) $-->$ as(Count), bs(Count), cs(Count).


as(0) $-->$ [].

as(NewCnt) $-->$ [a], as(Cnt), {NewCnt is Cnt + 1}.


bs(0) $-->$ [].

bs(NewCnt) $-->$ [b], bs(Cnt), {NewCnt is Cnt + 1}.


cs(0) $-->$ [].

cs(NewCnt) $-->$ [c], cs(Cnt), {NewCnt is Cnt + 1}.



\textbf{Seperating Rules from the Lexicon}

This means eliminating all mention of individual words in the DCG and to record all info about individual words in a seperate lexicon.


\includegraphics{modular}

\end{document}